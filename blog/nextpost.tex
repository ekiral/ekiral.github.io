\documentclass[12pt]{amsart}
%PACKAGES

\usepackage{amsmath, amssymb, amsfonts, amsthm}

\usepackage{mathrsfs}
\usepackage{cancel}
\usepackage{graphicx}

%ENVIRONMENTS
\newtheorem{remark}{Remark}%[section]


%MATRIX GROUPS

\DeclareMathOperator{\SL}{SL}
\DeclareMathOperator{\GL}{GL}
\DeclareMathOperator{\SO}{SO}

%CALIGRAPHIC LETTERS
\renewcommand{\O}{\mathcal{O}}
\newcommand{\Qcal}{\mathcal{Q}}
\newcommand{\Dscr}{\mathscr{D}}

%BLACKBOARD BOLD
\newcommand{\A}{\mathbb{A}}
\newcommand{\R}{\mathbb{R}}
\newcommand{\Z}{\mathbb{Z}}
\newcommand{\C}{\mathbb{C}}
\newcommand{\Q}{\mathbb{Q}}
\renewcommand{\H}{\mathbb{H}}
\newcommand{\F}{\mathbb{F}}
\newcommand{\N}{\mathbb{N}}
\newcommand{\E}{\mathbb{E}}
\newcommand{\DKL}{\mathbb{D}}


%MATRIX DELIMITERS
\newcommand{\bsm}{\left(\begin{smallmatrix}}
\newcommand{\esm}{\end{smallmatrix}\right)}
\newcommand{\bpm}{\begin{pmatrix}}
\newcommand{\epm}{\end{pmatrix}}


%FLAG COLORS
\usepackage[usenames,dvipsnames]{xcolor}
\newcommand{\red}{\textcolor{red}}
\newcommand{\blue}{\textcolor{blue}}
\newcommand{\green}{\textcolor{PineGreen}}
\newcommand{\purple}{\textcolor{purple}}
\newcommand{\cyan}{\textcolor{cyan}}
\newcommand{\yellow}{\textcolor{yellow}}
\newcommand{\orange}{\textcolor{orange}}

%THEOREMS
\newtheorem{theorem}{Theorem}
\newtheorem{lemma}[theorem]{Lemma}
\newtheorem{corollary}[theorem]{Corollary}
\newtheorem{proposition}[theorem]{Proposition}
\newtheorem{rmk}[theorem]{Remark}
\newtheorem{definition}{Definition}

%Miscellaneous
\newcommand{\vtheta}{\boldsymbol{\theta}}
\DeclareMathOperator{\val}{val}
\renewcommand{\Re}{\operatorname{Re}}
\renewcommand{\Im}{\operatorname{Im}}
\newcommand{\inv}{^{-1}}
\renewcommand{\d}{\,\mathrm{d}}
\newcommand{\sums}{\sideset{}{^*}\sum}
\DeclareMathOperator{\Mat}{Mat}
\DeclareMathOperator{\diag}{diag}
\renewcommand{\th}{\textsuperscript{th}}
\renewcommand{\vec}[1]{\mathbf{\boldsymbol{#1}}}
\DeclareMathOperator*{\argmin}{\arg\,\min\,\,}
\newcommand{\calX}{\mathcal{X}}
\newcommand{\calY}{\mathcal{Y}}
\DeclareMathOperator{\Prob}{Prob}




\title{Learning Distributions on the Parameter Space}
\author{emk}
\date{2025/08/10}
\begin{document}

\maketitle

Let's learn a probability distribution on the space of parameters, so that whenever we sample from this distribution we get good model parameters, and not the same one over and over again either. 

We follow the notation of the \href{https://ekiral.github.io/blog/2025-06-05_notation.html}{setting} where we have a parametrized model $f$ with parameter space $\Theta$. With training data, we construct a loss function $\ell(\vec{\theta})$ and a loss contribution $\ell_i(\vec{\theta})$ for each labeled data point $(\vec{x}_i, y_i) \in \mathcal{X} \times \mathcal{Y}$. 

Classically people minimize the loss function via gradient methods and find\footnote{or at least give it the old college try} a $\vec{\theta}^* \in \Theta$ which then they use to generalize the predictions $f(\cdot, \vec{\theta}^*) : \mathcal{X} \to \mathcal{Y}$. That gradients are used assumes a Riemannian structure, i.e. have an inner product structure in the tangent planes, but that's a talk for another day.


\subsection*{The likelihood interpretation}

There is an interpretation of $\ell_i(\vec{\theta})$ as the negative log likelihood of observing  the label $y$ \emph{given} the input $\vec{x}$ and the parameter $\vec{\theta}$. This interpretation can make sense when the loss between $f(\vec{x}, \vec{\theta})$ and the true label $y$ is supposed to be a model of how the label data gets generated.  But regardless of  we can still see the consequences of such an interpretation. So our assumption is 
\[
	\ell_i(\theta) = -\log \Prob(y_i | \vec{x}_i, \vec{\theta}) = - \log \Prob(\vec{x}, y | \vec{\theta}) + \text{const.}
\] 
wrong.

\subsection*{Distributions on the parameter space} 
\subsection*{Entropy and $KL$-divergence} 


\end{document}
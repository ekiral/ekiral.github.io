\documentclass[12pt]{amsart}
%PACKAGES

\usepackage{amsmath, amssymb, amsfonts, amsthm}

\usepackage{mathrsfs}
\usepackage{cancel}
\usepackage{graphicx}

%ENVIRONMENTS
\newtheorem{remark}{Remark}%[section]


%MATRIX GROUPS

\DeclareMathOperator{\SL}{SL}
\DeclareMathOperator{\GL}{GL}
\DeclareMathOperator{\SO}{SO}

%CALIGRAPHIC LETTERS
\renewcommand{\O}{\mathcal{O}}
\newcommand{\Qcal}{\mathcal{Q}}
\newcommand{\Dscr}{\mathscr{D}}

%BLACKBOARD BOLD
\newcommand{\A}{\mathbb{A}}
\newcommand{\R}{\mathbb{R}}
\newcommand{\Z}{\mathbb{Z}}
\newcommand{\C}{\mathbb{C}}
\newcommand{\Q}{\mathbb{Q}}
\renewcommand{\H}{\mathbb{H}}
\newcommand{\F}{\mathbb{F}}
\newcommand{\N}{\mathbb{N}}
\newcommand{\E}{\mathbb{E}}
\newcommand{\DKL}{\mathbb{D}}


%MATRIX DELIMITERS
\newcommand{\bsm}{\left(\begin{smallmatrix}}
\newcommand{\esm}{\end{smallmatrix}\right)}
\newcommand{\bpm}{\begin{pmatrix}}
\newcommand{\epm}{\end{pmatrix}}


%FLAG COLORS
\usepackage[usenames,dvipsnames]{xcolor}
\newcommand{\red}{\textcolor{red}}
\newcommand{\blue}{\textcolor{blue}}
\newcommand{\green}{\textcolor{PineGreen}}
\newcommand{\purple}{\textcolor{purple}}
\newcommand{\cyan}{\textcolor{cyan}}
\newcommand{\yellow}{\textcolor{yellow}}
\newcommand{\orange}{\textcolor{orange}}

%THEOREMS
\newtheorem{theorem}{Theorem}
\newtheorem{lemma}[theorem]{Lemma}
\newtheorem{corollary}[theorem]{Corollary}
\newtheorem{proposition}[theorem]{Proposition}
\newtheorem{rmk}[theorem]{Remark}
\newtheorem{definition}{Definition}

%Miscellaneous
\newcommand{\vtheta}{\boldsymbol{\theta}}
\DeclareMathOperator{\val}{val}
\renewcommand{\Re}{\operatorname{Re}}
\renewcommand{\Im}{\operatorname{Im}}
\newcommand{\inv}{^{-1}}
\renewcommand{\d}{\,\mathrm{d}}
\newcommand{\sums}{\sideset{}{^*}\sum}
\DeclareMathOperator{\Mat}{Mat}
\DeclareMathOperator{\diag}{diag}
\renewcommand{\th}{\textsuperscript{th}}
\renewcommand{\vec}[1]{\mathbf{\boldsymbol{#1}}}
\DeclareMathOperator*{\argmin}{\arg\,\min\,\,}
\newcommand{\calX}{\mathcal{X}}
\newcommand{\calY}{\mathcal{Y}}
\DeclareMathOperator{\Prob}{Prob}




\newcommand{\calX}{\mathcal{X}}
\newcommand{\calY}{\mathcal{Y}}
\DeclareMathOperator{\Prob}{Prob}

\title{Some Notation}
\author{emk}
\begin{document}

\maketitle

Let $\calX \times \calY$ be the data space split along an input-label line. The hypothesis class is a collection of functions $f \in \mathcal{F}$
\[
	f : \calX \times \Theta \to \calY.
\]
For example, the hypothesis class could be a neural network with $P$ weights (and biases), then $\Theta = \R^P$ and $f(\vec{x}, \vec{\theta})$ would be the function defined by the network.

A \emph{solution} to the supervised learning problem could be a $\vec{\theta}^*$ (a set of parameters) such that $$f(\vec{x}_i, \vec{\theta}^*) \approx y_i$$
for all given (training) data points $\{(\vec{x}_i, y_i)\}_{i = 1}^N \subseteq \calX \times \calY$. The whole enterprise on the assumption that $f$ also predicts the label for yet unseen data; i.e. $f(\vec{x}_{\text{new}}, \vec{\theta}^*) \approx y_{\text{new}}$ for yet unseen data $(\vec{x}_{\text{new}}, y_{\text{new}})$ assumed to come from the same distribution as the training data in $\calX \times \calY$. Whether this is possible heavily depends on the hypothesis class $\mathcal{F}$. Those functions describable by a neural network and easily discoverable by the common gradient based algorithms have proven to work well for distributions of data we have collected and thrown at these machine learning models.

Finding this $\vec \theta^*$ is typically achieved by a variant of gradient descent on the loss function defined of the form
\begin{equation}\label{eq:lossfn}
	\ell(\vec{\theta)} = \frac{1}{N} \sum_{i = 1}^N \ell_i(\vec{\theta}) + R(\vec{\theta}) \quad \text{ with } \quad \ell_i(\vec{\theta}) = c(f(\vec{x}_i, \vec{\theta}), y_i),
\end{equation}
where $R :\Theta \to \R_{\geq 0 }$ is is called the regularizer, responsible for biasing the hypothesis space to simpler functions (usually meaning smaller norm parameters), and $c : \calY \times \calY \to \R_{\geq 0}$ is called the cost function, measuring how far a prediction $\hat y$ is from $y$ with  $c(\hat y, y)$ (usually $c(y, y) = 0$). 

\begin{itemize}
\item For classification problems (with $K$ classes) it is common to take $\mathcal{Y} =  \Delta^K = \{(p_1, \ldots, p_K) \in \R^K : p_i \geq 0, \sum_{i = 1}^K p_i = 1\}$ the probability $K$-simplex, and the cost function as the cross entropy loss $c(\hat y, y) = - \sum_{i = 1}^k y_i \log \hat y_i$.
\item It is possible to consider $c : \widehat{\calY} \times \calY \to \R_{\geq 0}$, for example when the labels $y_i$ were given as indices and not one-hot vectors. But unless there is a correspondence between $\widehat{\calY} \rightarrow  \calY$ we wouldn't be able to use $f: \calX \times \Theta \to \widehat{\calY}$ as a label-predictor.
\item For the regression problem, with $\calY = \R^{\text{out}}$ it is common to take the $L^2$-norm as the cost function $c(\hat y, y) = \|\hat y - y\|^2$.
\item The parameter space $\Theta$ needs to be a differentiable manifold and $\ell$ needs to be a differentiable function of $\vec{\theta}$ (at generic points) in order for the gradient methods to make sense. 
\item The most ubiquitous setting is when $\Theta = \R^P$ and we then use the gradient descent method  $\vec{\theta}^{\text{updated}} = \vec{\theta} - \alpha \nabla \ell(\vec \theta)$. 
\item On a \emph{not necessarily flat} manifold, we would need a retraction function
\[
	\qquad \qquad r_{\vec{\theta}} : T_{\vec{\theta}} \Theta \to \Theta  \text{ satisfying } r_{\vec{\theta}}(\vec{0}) = \vec{\theta} \text{ and } \frac{\d r_{\vec{\theta}}(t \vec{v})} {\d t}\bigg|_{t = 0} = \vec{v},
\]
or a partial retraction function $r_{\vec{\theta}}: U \to \Theta$ for a neighborhood $\vec{0} \in U \subseteq T_{\vec{\theta}}\Theta$. The raison d'\^{e}tre such a retraction is to be able to move on $\Theta$ starting from $\vec{\theta}$, in the direction of $\vec{v}\in T_\vec{\theta} \Theta$ and we update the values of the parameters by $\vec\theta^{\text{updated}} := r_\vec{\theta}(- \alpha \nabla \ell(\vec{\theta}))$, valid for small enough step size $\alpha>0$ in partial retractions. For $\Theta$ an open domain in a flat affine space, $r_{\vec{\theta}} (\vec{v}) = \vec{v} + \vec{\theta}$ does the job, and gives us the gradient descent for $\vec{v} = -\alpha \nabla\ell(\vec{\theta})$.
\item The data space $\calX$ does not technically need any structure more than a set, however in most circumstances it will be a differentiable manifold---in fact simply $\calX = \R^{\text{in}}$---and in the case of neural networks, $f(\vec{x}, \vec{\theta})$ is a differentiable function of $\vec{x}$, using which we can compute saliency of features.
\item In case of $\Theta = \R^P$, we have $T_\vec{\theta} \Theta \cong \R^P$ for all points $\vec{\theta}$ on the manifold, and in fact they can be all canonically identified with one another. That is why we can assume all the gradients $\nabla \ell(\vec{\theta})$ are living in the same vector space, and why it is meaningful to accumulate gradients from past steps such as in momentum.
\item If furthermore the the tangent spaces $T_{\vec{\theta}}$ have distinguished bases---for example the directions aligned with weights is distinguished from the point of view of the model architecture---then fixing such a basis one can also use componentwise operations on the gradient vectors such as the squaring, division, taking squareroots etc.\ used in Adam.
\end{itemize}

\subsection*{Loss as log-likelihood} 

There is an interpretation of $\ell_i(\theta)$ as the negative log likelihood of observing  the label $y$ given the input $\vec{x}$ and the parameter $\vec{\theta}$. This interpretation makes sense when $f(\vec{x}, \vec{\theta}$ is supposed to be a model of how the label data gets generated.  But we can still see the consequences of such an interpretation. So our assumption is 
\[
	\ell(\theta) = -\log \Prob(y | \vec{x}, \vec{\theta}) = - \log \Prob(\vec{x}, y | \vec{\theta}) + \text{const.}
\] 

\subsection*{Distributions on the parameter space} 
\subsection*{Entropy and $KL$-divergence} 


\end{document}